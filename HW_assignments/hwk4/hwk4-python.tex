\documentclass[11pt]{article}    % stand alone latex
\usepackage{graphicx,epsfig}
\usepackage{listings}
\usepackage{enumerate}

%%%%%%%%%%% LAYOUT  %%%%%%%%%%%%%%%%%%%%%%%%%%%%
% LTSA : at least 1 inch (2.5 cm) margins on all sides; 
\topmargin  -0.03 in 
%\topmargin  1.0in
\oddsidemargin  0.05 in           % this give roughly 1.0 in left margin
%\evensidemargin 1.00 in
\textwidth  6.4 in  		% normal width 6 in        
\textheight 9.2 in 		%normal 22.86cm (9 in) high 
\headsep  0.05 in        % space between markboth and start of text
\footskip 0.25 in
%\footheight 0.25in
\pretolerance=500   		%allow hyphenation
\tolerance=500	    		%allow hyphenation
\rightskip=0pt   		% allow justification
%\parskip 10pt
%%%%%%%%%%%%%%%%%% DEFINITIONS %%%%%%%%%%%%%%%%%%%%%%%%
\def\bb{$B$-band}
\def\vb{$V$-band}
\def\rb{$R$-band}
\def\ib{$I$-band}
\def\zb{$z$-band}
% units 
\def\gtsim{\lower.5ex\hbox{$\buSildrel > \over\sim$}}
\def\ltsim{\lower.5ex\hbox{$\buildrel < \over\sim$}}
\def\arcsec{^{\prime\prime}}
\def\arcmin{^\prime}
% $1 \farcs 25$  for 1''.25
% $11\farcm 3 $  for  11'.3
\def\farcm{\hbox{$.\!\!^{\prime}$}} 
\def\farcs{\hbox{$.\!\!^{\prime\prime}$}} 
\def\farcss{\hbox{$.\!\!^{s}$}} 
\def\deg{{$^\circ$}}
\def\sun{\mbox{$_\odot$}}
\def\kms{~km~s$^{-1}$}
\def\jyb{~Jy~beam$^{-1}$}
\def\jybkms{~Jy~beam$^{-1}$ km~s$^{-1}$}
\def\mjyb{~mJy~beam$^{-1}$}
\def\ra#1#2#3{#1$^{\rm h}$#2$^{\rm m}$#3$^{\rm s}$}
\def\dec#1#2#3{$#1^\circ#2'#3''$}
\def\e#1{\underline{#1}}  % external PI is underlined in tables
\def\etal{{et al.~}}
\def\hst{{\it HST}}
\def\chandra{{\it Chandra}}
\def\sirtf{{\it SIRTF}}
%
%  journal reference
\def\apjl{ApJL}
\def\apj{ApJ}
\def\apjs{ApJS}
\def\mnras{MNRAS}
\def\araa{ARAA}
\def\aj{AJ}
\def\aap{A\&A}
\def\aaps{A\&A Suppl.}
\def\nat{Nature}
\def\pasp{PASP}
% chemical
\def\h2{H$_{\rm 2}$}
\def\co{CO}
\def\co1{CO~($J$=1--0)}
\def\tco{$^{13}$CO}
\def\tco1{$^{13}$CO~($J$=1--0)}
\def\ceo{C$^{18}$O}
%\def\hf{H$_2$ $\nu=0$--$1$ S(1)}

\def\hii{H{\sc ii}}
%\def\hii{\hbox{H{ }{\sevenrm II}{ }}}
\def\brg{Br~$\gamma$}
\def\water{H$_2$O}
\def\Ha{\,{\rm H\alpha}}
\def\vlsr{$v_{\rm LSR}$}
%
%
%%%%%%%%%%%%%%%%%%%%%%%%%%%%%%%%%%%%%%%%%%%%%%%%%%%%%%%%%%%%%%%%%%
\begin{document}
% PAGE STYLE FOR TABLE OF CONTENTS 
%\pagestyle{plain}    		
%\pagenumbering{roman}		% roman for Table of contents
%\tableofcontents
%\listoffigures
%\listoftables
%\clearpage
%
%  PAGE STYLE FOR TEXT 
%\pagestyle{empty}         	% suppress page numbering 
%\pagestyle{plain}   		% make page numbers at the bottom center
% \pagestyle{myheadings}
\pagenumbering{arabic}
%\markboth{}{Project Description}
\setcounter{page}{0}
\pagenumbering{arabic}	 	% arabic for main text 
\setcounter{section}{0}
%%%%%%%%%%%%%%%%%%%%%%%%%%%%%%%%%%%%%%%%%%%%%%%%%%%%%%%%%%%%%%%%%%

\begin{center}
\Large
{\bf Astro 376: Python Assignment 2 (Homework 4)\\
Exploring Star Formation and Extreme Systems At Early Epochs!}\\
\vspace{4 mm}
\large
{\bf (Assigned on Tu Apr 13, 2021. Due by 11:59pm CT on Tu Apr 20, 2021)}

\end{center}

% 1) Note by SJ in Oct 2013
%   Hwk 3 asked students to do a) b) c) below. This homework 4 also 
%   asks them to also do steps  d)  and  e)
%   - a) se readcol to read large formatted catalogs;  
%   - b) to extract a subset of array indices whose elements fulfill conditions
%        based on redshift, masses, SFR-UV, SFR-IR) bu using logical variables in  
%         conditional operations  (e.g., xcon1= photz ge 0.1;l xcon2 = photz le 1.0; 
%        xcon3 = stmass ge 1e10, etc) 
%   - c) to write formatted catalogs 
%   - d)  perform basic statistics 
%   - e)  make plots (scatter plots, histograms, etc) 
% 
%   2)  Before giving  out  this homework put 
%    /data1/proust/sj/azur1/classes/a376-fa12/hwk4-files/hwk4-catalog.txt
%    to 
%    /home/astro/exg/sj/www/a376-fa12/secure/hwk4/hwk4-catalog.txt
% 
%  3)  When giving out  this homework, explain or specify the following:
%   a) GO OVER ITEMS TO HAND IN : Remind them they  need to hand in 
%           -- hardcopy of your program with your name on it and 
%                relevant parts labelled  "Code for part 1",  "Code for
%                part 2" ,etc
%          --  all items requested in parts 1 to 5 (hardcopies)
%         AND WE RESERVE THE RIGHT TO INSPECT ELECTRONIC VERSION
%        OF THE PROGRAM IN THEIR hwk3 subdirectory     
%      
%    b) This is a hwk, so cannot tell you how to do it. We can only
%         give you a tip of the form `` look at section xx in your
%          IDL tutorial'' or ``review the function where''
%
%    c) The catalog you have in this homework is  TREASURE TROVE OF 
%        OF INFORMATION THAT TOOK ASTRONOMERS 5 YEARS TO UP TOGETHER
%        USING SOME OF THE MOST CUTTING EDGE SURVEYS DONE TO DATE.
%         IT TOOK THREE MAJOR SURVEYS OF GALAXIES FROM
%         THE GROUND (COMBO-17), ACS GEMS survey, and SPITZER SURVEY 
%         AND A LOT OF MODELING TO GET  M*, SFR_uv, SFR_ir, ETC 
%
%     d)  - Part 1:  If a catalog is written out with the following format 
%           format ='(G,1x,G,1x,A,1x,G)
%     then to read it in with readcol, you need to use format 
%            format ='(G,F,A,F)    --> replace F by G, and remove 1x
% 
%  e)  Part 3 to 5:  plotting log(x) is not same as plotting x on log scale
%      Part 3 to 5 : how to label with Greek symbols like Mo, or superscript    yr-1
%      Part 5: explain legend 
%


\begin{center}
\vspace{3.0 mm}
\noindent
{\bf{ Overview of Assignment}} 
%{\bf{Instructions}} 
\end{center}

\begin{itemize}
\item 
\noindent
In this homework, you will be applying what you have learnt in the
Python tutorial over the last few weeks, and going beyond the tasks
you did in homework 3.  For some questions, you will need
to go beyond the Python tutorial and consult online help. 
We also assume that by now, you are comfortable with basic Mac 
OSX/Linux commands and some text editor of your choice (like
Aquamacs or Vi). Recall that you can let Terminal choose a text editor
for you if you type `{\tt{open file}}'. If you need a refresher, please consult
the Mac OSX/Linux tutorial that you practiced in class earlier.
The number of points for each question is indicated in brackets,
and the total score is 100 points.

\item 
\noindent
Your main task is to write an Python program, which does the following: 

\vspace{-3mm}
\begin{itemize}
\item 
\noindent
Reads a  very large formatted catalog containing properties
of galaxies (e.g., their redshifts, stellar masses,  and rates of star
formation) out  to early epochs  {\it {when the Universe was less than 
half of its present age}}.
These properties were  derived using some of the most
cutting-edge surveys of galaxies conducted with NASA's 
{\it {Hubble Space Telescope}}  and   Spitzer Space Telescope,
as well as  ground-based observations at many different wavelengths.

\item 
\noindent 
Uses conditional operations on array indices to extract elements 
that satisfy specific criteria  (e.g., specific redshift ranges,
masses,  etc).

\item  
\noindent 
Performs specific operations and basic statistical analyses on select 
elements.


\item  
\noindent 
Produces different graphical representations of the results, including 
scatter plots, and  histograms.

\item  
\noindent 
Produces formatted output catalogs

\end{itemize}
\end{itemize}


\begin{center}
\vspace{3.0 mm}
\noindent
{\bf{ Detailed Instructions}}
\end{center}

\begin{enumerate}[(a)]

\item 
\vspace{2.0 mm}
\noindent
Create a sub-directory called `hwk4'  in your python directory and change 
(cd) to this directory. 
Retrieve the file {\it `hwk4-catalog.txt'}  by typing in a {\bf single} line 
\begin{lstlisting}
 curl -u cosmos:ast376web www.as.utexas.edu/~sj/a376-sp21/secure/hwk4/
 hwk4-catalog.txt -o hkw4-catalog.txt
\end{lstlisting} 

\item 
\noindent
Put all the files that you create for this homework into the 
sub-directory  `$\sim$/python/hwk4'.
{\it Your grade will be based  on the hardcopy of the homework that 
you hand in, as well as any electronic files you might be asked to 
produce in this directory.}



\item 
\noindent
Please write a single Python program 
to do all the questions in this homework and put in 
{\it {short comments within the program to indicate which part of the code 
addresses which question.}} For instance, the comments might read `Code
for question 1',  `Code for  question 2', etc. 
{\it{ Please name your Python program    `hwk4-myname.py', where  `myname' is  
your name  (e.g.,   hwk4-jogee-shardha.py.)}}  
{\bf{Please submit your Python file with your name on it on
Canvas, along with the other items requested in questions 1 to 8.}}


\item 
\noindent
{\bf{
For full credit, you must use the correct  way to program
in Python.}} Getting to the right answer using a bad method will 
only get you  partial credit. Note that copying the programs of your 
fellow students is considered as {\bf cheating} and will lead to a ``Fail'' grade for both
the person who copies the program and the person who allow 
their program to be copied.
\end{enumerate}


\begin{enumerate}

\vspace{1mm}
\item
\noindent
The catalog  `hwk4-catalog.txt' has information on a subset of 
galaxies, which have been observed with the optical 
Advanced Camera for Surveys (ACS) on the {\it Hubble Space Telescope}
and the mid-infrared MIPS  instrument at a wavelength of 24 microns aboard 
the {\it Spitzer Space Telescope}.  In addition, observations from 
ground-based telescopes from the ultra-violet (UV) to near-infrared (NIR) 
have also been taken.  Read through the header information in the catalog
to understand the different quantities that have been recorded for 
each galaxy. 

In this question, `log' denotes
the logarithm to the base 10,  $M$ is the stellar mass in unit of  
$M_{\odot}$, and SFR$_{\rm UV}$ is the unobscured
UV-based star formation rate  in units of $M_{\odot}$  yr$^{-1}$,
determined from UV data.
Write a Python program, 
which makes a plot of $\log$(SFR$_{\rm UV}$) versus $\log(M$), for 
those galaxies with photometric redshift $z$ in the range 
$0.47 < z \le 0.62$, and  stellar masses 
$M \ge$~$5 \times  10^{9}$ $M_{\odot}$. 
This redshift range corresponds to lookback times of 
4 to 5 billion years, an epoch when the Universe was 9.7 to 8.7 
billion years old or about two-thirds of its present age. 
Plot the galaxies as black triangles or black stars.
Label the x-axis with  `$\log(M/M_{\odot}$)', and the y-axis with
`$\log$(SFR$_{\rm UV}$/$M_{\odot}$  yr$^{-1}$)'.
Add a legend labeled `z =0.47--0.62' in one corner of the plot.
Save a copy of the figure in PDF format and turn it in.
 {\bf [25 points]}

\vspace{1mm}
\item
\noindent
For the galaxies selected in part (1), use Python to determine the
minimum, maximum, and median UV-based star formation rate in unit of 
$M_{\odot}$ yr$^{-1}$. You can do this by extending the Python program
in part (1) or writing a new program. We recommend the former method.
Submit a copy of your answers in a text file. 
 {\bf [10 points]}.
 

\vspace{1mm}
\item
\noindent 
The UV data  traces the unobscured star formation, namely star
formation that is not obscured by dust. Thus, the UV-based 
star formation rate
is a lower limit to the true total star formation rate. 
The  {\it Spitzer} 24 micron
observations trace the obscured part of the star formation and yield
the infrared-based obscured star formation rate, SFR$_{\rm IR}$. 
Unfortunately, only galaxies with fairly large infrared luminosities
are detected by the   {\it Spitzer} 24 micron observations, and 
SFR$_{\rm IR}$ are only available for those systems. In these cases, 
the total star formation rate, SFR$_{\rm tot}$, can  be calculated 
as (SFR$_{\rm IR}$ +  SFR$_{\rm UV}$). The value of the  `whmips'
parameter in  the catalog indicates whether or not there was a successful 
{\it Spitzer} 24 micron detection.


\noindent 
For galaxies with a successful {\it Spitzer} 24 micron detection, 
a photometric redshift $z$ in the range  $0.47 < z \le 0.62$, and 
stellar masses  $M \ge$~$5 \times  10^{9}$ $M_{\odot}$, write 
a Python program, which  makes a plot of  $\log$(SFR$_{\rm tot}$) 
versus $\log(M$).
Plot galaxies with $SFR_{\rm tot} \ge 3.0 \  M_{\odot}$ yr$^{-1}$ (i.e., 
with star formation rates higher than that of our Milky Way) as red stars, 
and the remaining galaxies as black stars. 
Label the x-axis with  `$\log(M/M_{\odot}$)', and the y-axis with
`$\log$(SFR$_{\rm tot}$/$M_{\odot}$  yr$^{-1}$)'.
Add a legend label for both data sets. Label the red stars as: `SFR$_{\rm tot} \geq 3.0 M_{\odot}$ yr$^{-1}$'
and label the black stars as `SFR$_{\rm tot} < 3.0 M_{\odot}$ yr$^{-1}$.
Save a copy of the figure in PDF format and turn it in.
{\bf [10 points]}.


\vspace{1mm}
\item
\noindent
For the galaxies selected in part (3), use Python to determine the
minimum, maximum, and median value of the total star formation 
rate (SFR$_{\rm tot}$), as well as the median value of the
UV-based star formation rate (SFR$_{\rm UV}$), 
in units of $M_{\odot}$ yr$^{-1}$. 
For these galaxies, what is the ratio $R$ of 
(median SFR$_{\rm tot}$/median  SFR$_{\rm UV}$)?
This ratio $R$ is a measure of how obscured the star formation sites
are.
Submit your answer in a text file. 
{\bf [10 points]}.



\vspace{1mm}
\item
\noindent
In addition to inspecting the mean or median value of the 
star formation rates, one can glean important conclusions from 
the distribution of star formation rates. 
For the galaxies selected in part (3), use Python to  make a plot that 
overlays two normalized histograms: a histogram of 
SFR$_{\rm tot}$ in  solid lines,  and  a histogram of SFR$_{\rm UV}$ 
in dotted lines.  
Label the y-axis with `Fraction of Galaxies' and the x-axis with 
`SFR ($M_{\odot}$  yr$^{-1}$)'.  
In one corner of the plot, add a legend to show that the solid
line represents SFR$_{\rm tot}$ , while the dotted line represents  
SFR$_{\rm UV}$. 
Save a copy of the figure in PDF format.
{\bf [15 points]}.


\vspace{1mm}
\item
\noindent
Our Galaxy, the Milky Way currently has a stellar mass of 
$4.3 \times  10^{10}$ $M_{\odot}$ and a total star formation rate 
of $\sim$ $3 M_{\odot} yr^{-1}$. 
Among the galaxies selected in part (3), 
how many systems are already more massive than the Milky way, 
4 to 5 billion years ago? 
Among the galaxies selected in part (3), 
how many systems had a SFR$_{\rm tot}$ exceeding that of 
the Milky Way, 4 to 5 billion years? 
Submit your answer in a text file. 
{\bf [10 points]}.

\vspace{1mm}
\item
\noindent
Among the galaxies selected in part (3), extract the 30 galaxies with
the largest stellar mass, and write out their properties (listed in  the 
original catalog  `hwk4-catalog.txt')  into a file called 
``list.massive.xxxxx.txt'', where xxxx is your last name. Be
sure to save the data as strings if using numpy.
Submit a copy of this file and of your Python program.
{\bf [10 points]}.


\vspace{1mm}
\item
\noindent
Repeat step (7), but extract the  30 galaxies with the largest 
total star formation rate, and call the file ``list.highest-sfr.xxxxx.txt'', 
where xxxx is your last name.
Submit a copy of this file and of your Python program.
{\bf [10 points]}.

\end{enumerate}

\vspace{2mm}

\noindent
{\textbf{For this assignment, you will submit the following:}}
\begin{itemize}
	\item A copy of your Python (.py) script file. If you created multiple script files to solve the HW,
		consolidate all of them into one code or submit them separately. If you choose the latter, please
		state clearly at the top of each .py file which part of the HW that script file is answering.
	\item A compressed .tar.gz file containing your figure outputs, in PDF format, for questions 1, 3, and 5.
	\item A text file containing your answers to questions 2, 4, and 6. The format of this file is not important, 
		we just need to see the answers written down somewhere.
	\item Two separate output catalogs corresponding to questions 7 and 8.
\end{itemize}

\noindent
We will review your python (.py) script files when we grade your submission, however, you should not expect us to run the code to obtain the answers to questions 2, 4, and 6.

\begin{center}
END OF ASSIGNMENT
\end{center}

\end{document}


latex hwk4-idl.tex
dvips -P pdf -o hwk4-idl.ps hwk4-idl
ps2pdf13  hwk4-idl.ps  hwk4-idl.pdf
or

pdflatex hwk4-idl.tex
